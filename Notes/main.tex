\documentclass[aps,prd,amsmath,superscriptaddress,floatfix,nofootinbib]{revtex4-2}

% math packages
\usepackage{mathrsfs}
\usepackage{amsfonts}
\usepackage{amsmath}
\usepackage{amssymb}
\usepackage{bm}
\usepackage{slashed}
\usepackage{physics}

\newcommand{\diff}[1]{{\rm d} #1}
\newcommand{\APV}{A_{\rm PV}}

% float packages
\usepackage{graphicx}
\graphicspath{{./figures}}
\usepackage[caption=false]{subfig}

% miscellaneous
\usepackage{siunitx}
\usepackage{xcolor}
\usepackage[colorlinks=true,citecolor=green,linkcolor=blue,urlcolor=blue]{hyperref}

% ref macros
\newcommand{\eref}[1]{Eq.~(\ref{eq:#1})}
\newcommand{\erefs}[2]{Eqs.~(\ref{eq:#1})--(\ref{eq:#2})}
\newcommand{\fref}[1]{Fig.~\ref{fig:#1}}
\newcommand{\frefs}[2]{Figs.~\ref{fig:#1}--\ref{fig:#2}}
\newcommand{\sref}[1]{Sec.~\ref{sec:#1}}
\newcommand{\ssref}[1]{Sec.~\ref{ss:#1}}
\newcommand{\sssref}[1]{Sec.~\ref{sss:#1}}
\newcommand{\tref}[1]{Table~\ref{tab:#1}}

\begin{document}

\title{Notes on PVDIS}
\author{Richard Whitehill}
\date{\today}
\maketitle

\section{Overview of Deep-Inelastic Scattering (DIS)}
\label{sec:overview-of-deep-inelastic-scattering}

The deep-inelastic scattering process refers generally to the lepton-nucleon collision $\ell N \rightarrow \ell' X$, where $X$ is the unobserved hadronic final state particle(s).
Since the original experiments at SLAC, DIS has been one of the most fruitful methods by which we have learned about the structures of the proton and neutron, and it will continue to be so for the foreseeable future, especially given the iminent operation of the Electron-Ion Collider (EIC) at Brookhaven national lab, which will give unprecedented access to a larger kinematic phase space.

For these notes, we restrict our attention to neutral-current processes, and in particular, we only consider contributions from processes that enter at first-order in boson exchanges between the lepton and nucleon (as shown in \fref{DIS-one-photon}).
In doing so, we have the following kinematic definitions:
\begin{itemize}
    \item $\displaystyle \nu = \frac{P \cdot q}{M} = E - E'$ is the energy lost by the lepton in the nucleon rest frame.
    \item $Q^2 - -q^2 = -(\ell - \ell')^2$ is the virtuality of the virtual photon.
    \item $\displaystyle x = \frac{Q^2}{2 P \cdot q} = \frac{Q^2}{2 M \nu}$ is the typical Bjorken scaling variable.
    \item $\displaystyle y = \frac{P \cdot q}{P \cdot \ell} = \frac{\nu}{E}$ is the inelasticity variable.
    \item $\displaystyle W^2 = (P + q)^2 = M^2 + Q^2 \Big( 1 - \frac{1}{x} \Big)$ is the invariant mass of the hadronic final state.
    \item $\displaystyle s = (P + \ell)^2 = M^2 + \frac{Q^2}{xy}$ is the squared center of mass energy of the lepton-nucleon system.
\end{itemize}
Note that DIS is characterized by the following kinematic constraints: $Q^2, W^2 \gg M^2$.

\begin{figure}[h!b]
\begin{center}
    \includegraphics[width=0.5\paperwidth]{./figures/DIS_generic.pdf}
    \caption{Generic feynman diagram representing the DIS process in the one-``photon'' exchange approximation.}
    \label{fig:DIS-one-photon}
\end{center}
\end{figure}

Using Fermi's Golden rule, we can write the cross section for this process (summing and integrating over the hadronic final states) as
\begin{eqnarray}
    \label{eq:xsec-general}
    \diff \sigma = \frac{1}{4 P \cdot \ell} \frac{\diff^3 {\bm \ell'}}{(2\pi)^{3} 2E'} \sum_{X} \int \Big( \prod_{i \in X} \frac{\diff^3 {\bm P_{i}}}{(2\pi)^3 2E_{i}} \Big) (2\pi)^{4} \delta^{(4)}(P + q - P_{X}) |{\cal M}|^2 
,\end{eqnarray}
Rewriting, we find
\begin{eqnarray}
    \label{eq:xsec-rewritten}
    \frac{\diff \sigma}{\diff E' \, \diff \Omega} = \frac{1}{2(s-M^2)} \frac{E'}{2(2\pi)^3} \sum_{X} \int \diff \Phi_{X} (2\pi)^{4} \delta^{(4)}(P+q-P_{X}) |{\cal M}|^2
.\end{eqnarray}

At this juncture, we analyze the form of the scattering amplitude.
The full amplitude is given as a sum of contributions from the processes mediated by a virtual photon and $Z$-boson, respectively,
\begin{eqnarray}
    \label{eq:DIS-amp}
    {\cal M} = {\cal M}_{\gamma} + {\cal M}_{Z}
,\end{eqnarray}
where $\displaystyle {\cal M}_{\gamma(Z)} = \bra{\ell',s'} j_{\mu}^{\gamma(Z)} \ket{\ell,s} \frac{g^{\mu\nu}}{Q^2} \bra{X} J_{\nu}^{\gamma(Z)} \ket{P,S} = \frac{1}{Q^2} \bra{\ell',s'} j_{\mu}^{\gamma(Z)} \ket{\ell,s} \bra{X} J^{\mu}_{\gamma(Z)} \ket{P,S}$, where $j_{\mu}^{\gamma(Z)}$ and $J_{\mu}^{\gamma(Z)}$ are electromagnetic (weak) lepton and nucleon currents.
Note that the factor $g^{\mu\nu}/Q^2$ is the virtual boson propagator, which is exact when the boson is a photon and approximate when the virtual particle is a $Z$-boson ($Q^2 \ll M_{Z}^2$).
The squared amplitude is then
\begin{eqnarray}
    \label{eq:DIS-sq-amp}
    |{\cal M}|^2 = |{\cal M}_{\gamma}|^2 + {\cal M}_{\gamma}^{*}{\cal M}_{Z} + {\cal M}_{Z}^{*}{\cal M}_{\gamma} + |{\cal M}_{Z}|^2
.\end{eqnarray}
Using the current matrix elements above, we can separate the leptonic and hadronic contributions to the cross section as a tensor product:
\begin{eqnarray}
    \label{eq:xsec-lep-had-tensors}
    \frac{\diff \sigma}{\diff E' \, \diff \Omega} = \frac{1}{2(s-M^2)} \frac{E'}{2(2\pi)^3} \frac{e^4}{Q^{4}} \sum_{i} [\eta_{i} L_{\mu\nu}^{i} (4\pi W^{\mu\nu}_{i})]
,\end{eqnarray}
where $i \in \{ \gamma, \gamma Z, Z \}$, $L_{\mu\nu}^{i}$ is the leptonic tensor, $W^{\mu\nu}_{i}$ is the hadronic tensor (where the factor of $4\pi$ is a normalization factor introduced by convention), and $\eta_{i}$ collects kinematic factors from the squared matrix elements.
We write down the kinematic factors $\eta_{i}$ with the corresponding lepton tensor $L_{\mu\nu}^{i}$:
\begin{itemize}
    \item $\eta_{\gamma} = 1$ and $L_{\mu\nu} = 2(\ell^{\mu} \ell'_{\nu} + \ell'_{\mu} \ell_{\nu} - g_{\mu\nu} \ell \cdot \ell' - i \lambda_{\ell} \epsilon_{\mu\nu\alpha\beta} \ell^{\alpha}\ell'^{\beta})$
    \item $\displaystyle \eta_{\gamma Z} = \Big( \frac{G_{F}M_{Z}^2}{2\sqrt{2} \pi \alpha} \Big) \Big( \frac{Q^2}{Q^2 + M_{Z}^2} \Big)$ and $L_{\mu\nu}^{\gamma Z} = -(g_{V}^{\ell} - \lambda g_{A}^{\ell})L_{\mu\nu}^{\gamma}$
    \item $\eta_{Z} = \eta_{\gamma Z}^2$ and $L_{\mu\nu}^{Z} = (g_{V}^{\ell} - \lambda g_{A}^{\ell})^2 L_{\mu\nu}^{\gamma}$
\end{itemize}
Note that $\displaystyle g_{V}^{\ell} = g_{A}^{\ell} - 2 Q_{\ell} \sin^2{\theta_{W}}$ and $\displaystyle g_{A}^{\ell} = \frac{1}{2} \frac{Q_{\ell}}{|Q_{\ell}|}$ are the vector and axial weak couplings ($\theta_{W}$ is the weak mixing angle) for lepton $\ell$ (with charge $Q_{\ell}$), respectively, and $\lambda_{\ell}$ is the helicity of the lepton (assuming longitudinal polarization).
Finally, we write down the definition of the hadronic tensor as
\begin{subequations}
    \label{eq:hadron-tensor-def}
    \begin{eqnarray}
        W^{\mu\nu}_{\gamma} = \frac{1}{4\pi} \sum_{X} \int \diff \Phi_{X} \delta_{\rm PS}^{(4)} \bra{P,S} J^{\mu \, \dagger}_{\gamma} \ket{X} \bra{X} J^{\nu}_{\gamma} \ket{P,S}
    \end{eqnarray}
    \begin{align}
        W^{\mu\nu}_{\gamma Z} = \frac{1}{4\pi} \sum_{X} \int \diff \Phi_{X} \delta_{\rm PS}^{(4)} \big[ &\bra{P,S} J^{\mu \, \dagger}_{\gamma} \ket{X} \bra{X} J^{\nu}_{Z} \ket{P,S} \notag \\
                                                                                                        &+ \bra{P,S} J^{\mu \, \dagger}_{Z} \ket{X} \bra{X} J^{\nu}_{\gamma} \ket{P,S} ]
    \end{align}
    \begin{eqnarray}
        W^{\mu\nu}_{Z} = \frac{1}{4\pi} \sum_{X} \int \diff \Phi_{X} \delta_{\rm PS}^{(4)} \bra{P,S} J^{\mu \, \dagger}_{Z} \ket{X} \bra{X} J^{\nu}_{Z} \ket{P,S}
    ,\end{eqnarray}
\end{subequations}
where $\delta^{(4)}_{\rm PS} = (2\pi)^{4} \delta^{(4)}(P+q-P_{X})$.
Generally, we can parameterize the hadronic tensor in terms of structure functions, ensuring that it obeys current conservation (i.e. $\partial_{\mu}J^{\mu} = 0 \Leftrightarrow q_{\mu}W^{\mu\nu} = 0$) and Lorentz covariance, as
\begin{eqnarray}
\label{eq:hadronic-tensor-structure-functions}
\begin{aligned}
    W^{\mu\nu} &= -\tilde{g}^{\mu\nu} F_{1}(x,Q^2) + \tilde{P}^{\mu}\tilde{P}^{\nu} \frac{F_{2}(x,Q^2)}{P \cdot q} - i \epsilon^{\mu\nu\alpha\beta} \frac{q^{\alpha}P^{\beta}}{2 P \cdot q} F_{3}(x,Q^2) \\
               &+ i\epsilon^{\mu\nu\alpha\beta} \frac{q^{\alpha}}{P \cdot q} \Big[ S^{\beta} g_{1}(x,Q^2) + \Big( S^{\beta} - \frac{S \cdot q}{P \cdot q} \Big) g_{2}(x,Q^2) \Big] \\
               &+ \frac{1}{P \cdot q} \Big[ \frac{1}{2} \Big( \tilde{P}^{\mu}\tilde{S}^{\nu} + \tilde{S}^{\mu}\tilde{P}^{\nu} \Big) - \frac{S \cdot q}{P \cdot q} \tilde{P}^{\mu}\tilde{P}^{\nu} \Big] g_{3}(x,Q^2) \\
               &+ \frac{S \cdot q}{P \cdot q} \Big[ \tilde{P}^{\mu}\tilde{P}^{\nu} \frac{g_{4}(x,Q^2)}{P \cdot q} - \tilde{g}^{\mu\nu} g_{5}(x,Q^2) \Big]
,\end{aligned}
\end{eqnarray}
where $\displaystyle \tilde{g}^{\mu\nu} = g^{\mu\nu} - \frac{q^{\mu}q^{\nu}}{q^2}$ and $\tilde{P}^{\mu} = \tilde{g}^{\mu\nu} P_{\nu}$.
Furthermore, we note that $S$ is the nucleon 4-vector with $S^2 = -M^2$ and $S \cdot P = 0$, and lastly, we note that $F_{j}(x,Q^2)$ and $g_{j}(x,Q^2)$ are the unpolarized and polarized\footnote{It should be noted that $W^{\mu\nu} = W^{\mu\nu}(P,q,S)$, and unpolarized structure functions appear in the sum $W^{\mu\nu}(P,q,S) + W^{\mu\nu}(P,q,-S)$ while polarized structure functions appear in the difference $W^{\mu\nu}(P,q,S) - W^{\mu\nu}(P,q,-S)$.} dimensionless Bjorken-scaling structure functions, respectively.

Now, we can change variables from $(E',\Omega)$ to $(x,y)$, which gives us the conventional DIS cross section as
\begin{eqnarray}
\label{eq:xsec-xy}
    \frac{\diff \sigma}{\diff x \, \diff y} = \frac{Q^2}{x} \frac{\pi}{E'} \frac{\diff \sigma}{\diff E' \, \diff \Omega} = \frac{2 \pi \alpha^2 y}{Q^4} \sum_{i} \eta_{i} L_{\mu\nu}^{i}W_{i}^{\mu\nu}
,\end{eqnarray}
where we have used the relation $e^2 = 4\pi\alpha$.
Note that $\diff \sigma = \diff \sigma(\lambda_{\ell}, S)$ depends on the helicity of the lepton (in the initial state) and spin of the nucleon.
We can therefore define cross sections that depend on exclusively on the spin of either the lepton or the nucleon.
When we refer to the polarization of a cross section, we explicitly refer to the polarization of the nucleon.
Therefore, the unpolarized cross section is given by
\begin{eqnarray}
\label{eq:upol-xsec}
\begin{aligned}    
    \frac{\diff \sigma}{\diff x \, \diff y} &= \frac{1}{2} \Bigg[ \frac{\diff \sigma(\lambda_{\ell},-S)}{\diff x \, \diff y} + \frac{\diff \sigma(\lambda_{\ell},S)}{\diff x \, \diff y} \Bigg] = \frac{2 \pi \alpha^2 y}{Q^{4}} \sum_{i} \eta_{i} L_{\mu\nu}^{i} W^{\mu\nu}_{U,i} \\
                                                &= \frac{4 \pi \alpha^2}{x y Q^2} \Bigg[ xy^2 F_{1}^{\rm NC}(x,Q^2) \Bigg( 1 - y - \frac{x^2 y^2 M^2}{Q^2} \Bigg) F_{2}^{\rm NC}(x,Q^2) + \lambda_{\ell} x \Bigg( y - \frac{y^2}{2} \Bigg) F_{3}^{\rm NC}(x,Q^2) \Bigg]
.\end{aligned}
\end{eqnarray}
where $W^{\mu\nu}_{U} = \frac{1}{2} [ W^{\mu\nu}(P,q,S) + W^{\mu\nu}(P,q,-S) ]$ and 
\begin{eqnarray}
\label{eq:NC-upol-structure-funcs}
    F_{j}^{\rm NC} = F_{j}^{\gamma} - \eta_{\gamma Z} (g_{V}^{\ell} - \lambda_{\ell} g_{A}^{\ell}) F_{j}^{\gamma Z} + \eta_{Z} (g_{V}^{\ell} - \lambda_{\ell} g_{A}^{\ell})^2 F_{j}^{Z}
.\end{eqnarray}
Finally, the polarized cross section is given by\footnote{We assume longitudinal polarization such that $S^{\mu} = M(0,0,0,\lambda_{n})$, where $\lambda_{n}$ is the nucleon's helicity, in the proton's rest frame, which gives $S \cdot \ell = - \lambda_{n} M E$ and $S \cdot \ell' = -\lambda_{n} M E' \cos{\theta}$, where $\theta$ is the angle of deflection of the lepton. Note that $E = Q^2/[2Mxy]$, $E' = E(1-y)$, and $\cos{\theta} = 1 - Q^2/[2E^2(1-y)]$}
\begin{eqnarray}
    \label{eq:pol-xsec}
    \begin{aligned}    
        \frac{\diff \Delta \sigma}{\diff x \, \diff y} &= \frac{1}{2} \Bigg[ \frac{\diff \sigma(\lambda_{\ell},-S)}{\diff x \, \diff y} - \frac{\diff \sigma(\lambda_{\ell},S)}{\diff x \, \diff y} \Bigg] = \frac{2 \pi \alpha^2 y}{Q^{4}} \sum_{i} \eta_{i} L_{\mu\nu}^{i} W^{\mu\nu}_{P,i} \\
        &= \frac{8 \pi \alpha^2}{x y Q^2} \Bigg[ -\lambda_{\ell} xy \Bigg( 2 - y - 2x^2y^2 \frac{M^2}{Q^2} \Bigg) g_1^{\rm NC} + 4 \lambda_{\ell} x^3 y^2 \frac{M^2}{Q^2} g_2^{\rm NC} \\
        &+ 2x^2y \frac{M^2}{Q^2} \Bigg( 1 - y - x^2y^2 \frac{M^2}{Q^2} \Bigg) g_3^{\rm NC} \\
        &- \Bigg( 1 + 2x^2y \frac{M^2}{Q^2} \Bigg) \Bigg( \Bigg[ 1 - y - x^2y^2 \frac{M^2}{Q^2} \Bigg] g_4^{\rm NC} \Bigg) + xy^2 g_5^{\rm NC} \Bigg]
    ,\end{aligned}
\end{eqnarray}
where $W^{\mu\nu}_{P} = \frac{1}{2} [ W^{\mu\nu}(P,q,S) - W^{\mu\nu}(P,q,-S) ]$ and $g^{\rm NC}_{j}$ are defined similarly as for the unpolarized neutral current structure functions in \eref{NC-upol-structure-funcs}.
Finally, before moving on, we define
\begin{eqnarray}
    \label{eq:ratio}
    R^{i} = \frac{\sigma_{L}}{\sigma_{T}} = \frac{F_{2}^{i}}{2xF_{1}^{i}} r^2 - 1
,\end{eqnarray}
where $r^2 = 1 + Q^2/\nu^2$ and $\sigma_{L}$ and $\sigma_{T}$ are the longitudinal and transverse photoabsorption cross sections.


\section{The Parton Model}
\label{sec:the-naive-parton-model}

In the previous section, we outlined the calculation of the hadronic tensor and differential cross sections at the structure function level.
In this section, we outline the calculation of these quantities at the parton level, using a factorization framework.
At large $Q^2$, we can write the hadronic tensor as a convolution of hard scattering amplitudes and parton distribution functions (PDFs)\footnote{Neglecting higher order $m_{q}/Q$ corrections}:
\begin{subequations}
\label{eq:factorizations}
\begin{eqnarray}
    \label{eq:upol-had-tens-factorization}
    W^{\mu\nu}_{U} = \sum_{i} \int_{0}^{1} \frac{\diff \xi}{\xi} \widehat{W}^{\mu\nu}_{U}(\hat{x},Q^2) f_{i/P}(\xi,\mu^2)
\end{eqnarray}
and
\begin{eqnarray}
    \label{eq:pol-had-tens-factorization}
    W^{\mu\nu}_{P} = \sum_{i} \int_{0}^{1} \frac{\diff \xi}{\xi} \widehat{W}^{\mu\nu}_{P}(\hat{x},Q^2) \Delta f_{i/P}(\xi,\mu^2)
,\end{eqnarray}
\end{subequations}
where $\widehat{W}^{\mu\nu}$ is the partonic tensor (defined as the partonic analogue to the leptonic tensor), $(\Delta)f_{i/N}$ is the (polarized) unpolarized PDF for a parton of type $i$ in nucleon $N$, $\xi$ is the parton momentum fraction defined by $p = \xi P$, $\hat{x} = x/\xi$ is the partonic analogue to Bjorken-$x$, and $\mu$ is the renormalization scale for the PDFs (usually set such that $\mu^2 = Q^2$).
In the following text, we are interested in computing the hadronic structure functions in this factorization framework.
It is noted that several components go into the computation of the factorization integrand and the extraction of the structure functions.
Since these notes are focused on the phenomenology of PVDIS, we make no attempt to discuss the determination of the PDFs.
We simply utilize those PDFs, which were obtained from previous global QCD analyses, assuming that these studies were competent.
In the next sections, we outline the calculation of the partonic tensor (\ssref{calculation-of-the-partonic-tensor}) and the determination of the hadronic structure functions (\ssref{determination-of-the-hadronic-structure-functions}).

\subsection{Calculation of the partonic tensor}
\label{ss:calculation-of-the-partonic-tensor}

The partonic structure tensor is calculated as the squared amplitude for the virtual compton scattering process, shown in \fref{virtual-compton-scattering}, as
\begin{eqnarray}
    \label{eq:partonic-tensor}
    \widehat{W}^{\mu\nu}_{i} = \sum_{\rm graphs} {\cal M}^{\mu \, \dagger}_{i} {\cal M}^{\nu}_{i}
.\end{eqnarray}
In this work, we compute the partonic tensor at leading order, where the leading order process is simply $\gamma^{*} (q) + q (p) \rightarrow q (p')$ (shown in \fref{virtual-compton-scattering-LO}), which gives
\begin{subequations}    
\label{eq:partonic-tensor-LO}
\begin{eqnarray}
    \widehat{W}^{\mu\nu}_{\gamma} = \int \diff \Pi \, \Bigg\{ \sum_{s'} e_{q}^2 [ \bar{u}(p',s') \gamma^{\mu} u(p,s) ]^{*}[ \bar{u}(p',s') \gamma^{\nu} u(p,s) ] \Bigg\}
\end{eqnarray}
\begin{eqnarray}
\begin{aligned}
    \widehat{W}^{\mu\nu}_{\gamma Z} = \int \diff \Pi \, \Bigg\{ \sum_{s'} e_{q} ( [ &\bar{u}(p',s') ( g_{V}^{i} - g_{A}^{i} \gamma^{5} ) \gamma^{\mu} u(p,s) ]^{*}[ \bar{u}(p',s') \gamma^{\nu} u(p,s) ]  \\
                                                                            &+ [ \bar{u}(p',s') \gamma^{\mu} u(p,s) ]^{*}[ \bar{u}(p',s') \gamma^{\nu} ( g_{V}^{i} - g_{A}^{i} \gamma^{5} ) u(p,s) ] ) \Bigg\}
\end{aligned}
\end{eqnarray}
\begin{eqnarray}
    \widehat{W}^{\mu\nu}_{Z} = \int \diff \Pi \, \Bigg\{ \sum_{s'} [ \bar{u}(p',s') \gamma^{\mu} ( g_{V}^{i} - g_{A}^{i} \gamma^{5} ) u(p,s) ]^{*}[ \bar{u}(p',s') \gamma^{\nu} ( g_{V}^{i} - g_{A}^{i} \gamma^{5} ) u(p,s) ] \Bigg\}
.\end{eqnarray}
Note that $e_{q}$ is the fractional charge of the quarks.
\end{subequations}
For the above tensors, note that
\begin{eqnarray}
    \label{eq:momentum-delta-func}
    \int \diff \Pi = \int \frac{\diff^3 {\bm p'}}{(2\pi)^3 2p'^{0}} \delta^{(3)}(p + q - p') = \int \diff^{4} p' \delta^{(4)}(p+q-p') \delta_{+}(p'^2)
,\end{eqnarray}
where $\displaystyle \delta_{+}(p'^2) = \delta_{+}([p+q]^2) = \frac{x}{Q^2}\delta(\xi - x)$ is the on-shell condition for the scattered parton, and\footnote{Recall the dirac adjoint: $\overline{\psi} = \psi^{\dagger} \gamma^{0}$}
\begin{eqnarray}
    \label{eq:generic-tensor-structure}
    \begin{aligned} 
        \sum_{s'} &[ \bar{u}(p',s') \gamma^{\mu} (v_1 - a_1 \gamma^{5}) u(p,s) ]^{*}[ \bar{u}(p',s') \gamma^{\nu} (v_2 - a_2 \gamma^{5}) u(p,s) ] \\
                  &= \sum_{s'} [ \bar{u}(p,s) (v_1 + a_1 \gamma^{5}) \gamma^{\mu} u(p',s') \bar{u}(p',s') \gamma^{\nu} (v_2 - a_2 \gamma^{5}) u(p,s) ] \\
                  &= \frac{1}{2}{\rm Tr} \Big[ \Big( 1 + \frac{\gamma^{5} \slashed{s}}{m_{q}} \Big) ( \slashed{p} + m_{q} ) ( v_1 + a_1 \gamma^{5} ) \gamma^{\mu} (\slashed{p}' + m_{q} ) \gamma^{\nu} ( v_2 - a_2 \gamma^{5} ) \Big] \\
                  &= 2 ( a_1 a_2 + v_1v_2 )[ 2 p^{\mu}p^{\nu} + p^{\mu}q^{\nu} + q^{\mu}p^{\nu}  - g^{\mu\nu} p \cdot q ] \\
                  &- 2 a_1v_2 [ 2s^{\mu}p^{\nu} - g^{\mu\nu} s \cdot q ] - 2 a_2v_1 [ 2p^{\mu}s^{\nu} - g^{\mu\nu} s \cdot q ] - 2 ( a_1v_2 + a_2v_1 ) [ s^{\mu}q^{\nu} + q^{\mu}s^{\nu} ] \\
                  &+ 2i\epsilon^{\mu\nu\alpha\beta} [ ( a_1v_2 + a_2v_1 ) p_{\alpha}q_{\beta} - 2 a_1a_2 s_{\alpha}p_{\beta} - ( a_1a_2 + v_1v_2 ) s_{\alpha}q_{\beta} ]
    .\end{aligned}
\end{eqnarray}

\begin{figure}[h!tb]
\begin{center} 
    \includegraphics[width=0.4\paperwidth]{DIS_parton_model.pdf}
    \caption{Generic feynman diagram for virtual compton scattering of partons from the exchanged photon in the DIS process. Note that the blue blob represents all the possible hard scattering (virtual compton scattering) diagrams that can be drawn, which contribute to the scattering process, and the red blob represents the soft part of the collinear factorization, which is captured by the PDFs.}
    \label{fig:virtual-compton-scattering}
\end{center}
\end{figure}

\begin{figure}[h!tb]
\begin{center} 
    \includegraphics[width=0.2\paperwidth]{DIS_LO.pdf}
    \caption{Leading order virtual compton scattering feynman diagram for the hard scattering part of \fref{virtual-compton-scattering}: $\gamma^{*}q \rightarrow q$.}
    \label{fig:virtual-compton-scattering-LO}
\end{center}
\end{figure}


\subsection{Determination of the hadronic structure functions}
\label{ss:determination-of-the-hadronic-structure-functions}

From the last section, we have the form of the electromagnetic, weak, and interference partonic tensors, and the connection between the hadronic and partonic tensors is captured in \eref{factorizations}.
The structure functions enter at the hadronic level as shown in \eref{hadron-tensor-def}.
Generically, we can express this form as a linear combination of structure functions\footnote{The argument is given here for the unpolarized and can be repeated for the polarized sector.}
\begin{eqnarray}
    \label{eq:hadronic-tensor-generic}
    W^{\mu\nu}_{U} = \sum_{j} a_{j}^{\mu\nu} F_{j}
.\end{eqnarray}

As with regular vectors, we can use 2nd-order projection tensors with the defining property that ${\cal P}_{\mu\nu}^{k} W^{\mu\nu}_{U} = F_{k}$.
For the unpolarized projectors, we can write a general projector as
\begin{eqnarray}
    \label{eq:general-projector}
    {\cal P}^{k}_{\mu\nu} = c_{g}^{k} {\cal P}^{g}_{\mu\nu} + c_{PP}^{k} {\cal P}^{PP}_{\mu\nu} + c_{\epsilon Pq}^{k} {\cal P}^{\epsilon Pq}_{\mu\nu}
,\end{eqnarray}
where ${\cal P}_{g}^{\mu\nu} = g^{\mu\nu}$, ${\cal P}_{PP}^{\mu\nu} = P^{\mu}P^{\nu}$, and ${\cal P}_{\epsilon Pq} = i \epsilon^{\mu\nu\alpha\beta}P_{\alpha}q_{\beta}$.
It is clear then from this setup that we can solve for the coefficients $c^{k}_{g,PP,\epsilon Pq}$ by imposing ${\cal P}_{\mu\nu}^{k} a_{j}^{\mu\nu} = \delta_{jk}$.
Doing so, gives the following projectors for the unpolarized structure functions
\begin{subequations}    
\label{eq:unpolarized-projectors}
\begin{eqnarray}
    {\cal P}_{1}^{\mu\nu} = -\frac{1}{2} {\cal P}_{g}^{\mu\nu} + \frac{2x^2}{4M^2x^2 + Q^2} {\cal P}_{PP}^{\mu\nu}
\end{eqnarray}
\begin{eqnarray}
    {\cal P}_{2}^{\mu\nu} = -\frac{Q^2 x}{4M^2x^2 + Q^2} \Big[ {\cal P}_{g} - \frac{12 x^2}{4M^2x^2 + Q^2} {\cal P}_{PP}^{\mu\nu} \Big]
\end{eqnarray}
\begin{eqnarray}
    \label{eq:label}
    {\cal P}_{3}^{\mu\nu} = -\frac{2x}{4M^2x^2 + Q^2} {\cal P}_{\epsilon Pq}
.\end{eqnarray}
\end{subequations}

For the calculations below, we will assume that $M^2 \ll Q^2$ such that the nucleon mass can be eglected.
Using this fact, putting everything into the factorization, and extracting the structure functions using the projectors above, we find the leading order expressions for the unpolarized structure functions:
\begin{align}
\label{eq:unpolarized-LO-structure-functions}
F_{1}^{[\gamma,\gamma Z,Z]}(x,Q^2) &= \frac{1}{2} \sum_{q} \Big[ e_{q}^2, \, 2e_{q}g_{V}^{q}, \, (g_{V}^{q})^2+(g_{A}^{q})^2 \Big] f_{q}^{+}(x,Q^2) \\
F_{3}^{[\gamma,\gamma Z,Z]}(x,Q^2) &= \phantom{\frac{1}{2}} \sum_{q} \Big[ 0, \, 2e_{q}g_{A}^{q}, \, 2g_{V}^{q}g_{A}^{q} \Big] f_{q}^{-}(x,Q^2)
,\end{align}
where we have defined $f^{\pm}_{q} = f_{q} \pm f_{\bar{q}}$.
Also, note that $F_{2} = 2xF_{1}$ (which is the Callan-Gross relation) and the sum over $q$ is a sum over quark flavors ($u,\, d, \, s, \, c, \, b$).\footnote{The top (anti-)quark does not contribute at the $\sqrt{s}$ we consider since it is so massive ($m_{t} \approx \SI{173}{\GeV}$)!.}

\section{Parity-Violating Asymmetry}
\label{sec:parity-violating-asymmetry}

Now that we have the expressions for the hadronic cross sections and structure functions, we may write down and compute the parity violating asymmetry $\APV$ over a wide-range of kinematics.
In particular, we are interested in the dependence of $\APV$ on the strange sea.
Current experimental scattering data do not allow global QCD analyses to constrain the strange sea well, but because $\APV$ depends directly on combinations of the strange and anti-strange PDFs (as will be shown below), it may be possible that this asymmetry serves as a good candidate for data input to future global global QCD analyses.

The conventional asymmetry is defined as the ratio below\footnote{We have defined $\diff \sigma_{\pm} = \diff \sigma(\lambda_{\ell} = \pm 1)$, where $\diff \sigma$ is the unpolarized cross section of \eref{upol-xsec}.}:
\begin{eqnarray}
    \label{eq:APV}
    \APV = \frac{\diff \sigma_{+} - \diff \sigma_{-}}{\diff \sigma_{+} + \diff \sigma_{-}} = \frac{G_{F} Q^2}{8 \pi \alpha} \Big[ a_{1}Y_{1} + a_{3} Y_{3} \Big]
,\end{eqnarray}
where
\begin{eqnarray}
    \label{eq:a13}
    a_{1} = 2 g_{A}^{\ell} \frac{F_{1}^{\gamma Z}}{F_{1}^{\gamma}} \, \mbox{ and } \, a_{3} = g_{V}^{\ell} \frac{F_{3}^{\gamma Z}}{F_{3}^{\gamma}}
.\end{eqnarray}
Additionally, the factors
\begin{align}
    \label{eq:Y13}
    Y_{1} &= \Bigg( \frac{1 + R^{\gamma Z}}{1 + R^{\gamma}} \Bigg) \frac{1 + (1-y^2) - \frac{y^2}{2} \Big[ 1 + r^2 - \frac{2r^2}{1 + R^{\gamma Z}} \Big]}{1 + (1-y)^2 - \frac{y^2}{2}\Big[ 1 + r^2 - \frac{2r^2}{1 + R^{\gamma}} \Big]} \\
    Y_{3} &= \Bigg( \frac{1 + R^{\gamma Z}}{1 + R^{\gamma}} \Bigg) \frac{1 - (1-y^2)}{1 + (1-y)^2 - \frac{y^2}{2}\Big[ 1 + r^2 - \frac{2r^2}{1 + R^{\gamma}} \Big]}
.\end{align}
In the leading order calculation we completed above, we had $F_2 = 2x F_1$, implying that $R = r^2 - 1$.
This gives $Y_{1} = 1$ and $Y_{3} = [1 - (1-y)^2]/[1 + (1-y)^2]$.
It should be noted that the expression recorded for $\APV$ in \eref{APV} is the leading-order term in the power series expansion of the general $\APV$ in $G_{F}/\alpha$ and $M_{Z} / Q$.






\end{document}
